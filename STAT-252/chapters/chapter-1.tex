\documentclass[../STAT-252-Notes.tex]{subfiles}
\begin{document}
\chapter{Review}
This is a review on all things from STAT 151 or STAT 1770 from the U of L. 

You should know what a population is, what a sample is, etc. You should know what a parameter is and what a statistic is in term of descriptive measure. 

\section{Dichotomous Tree on Types of Inferences Possible for Observation and Experiment}
This can be explained using a simple if statement
\begin{listing}[h]
\begin{minted}[linenos,numbersep=5pt,frame=lines,framesep=2mm]{cpp}
if (random_selection from the target population):
    Population inferences can be made
if (random assignment to treatment):
    Causal inferences can be made

\end{minted}
\caption{Inference Tree}
\label{lst:inference_tree}
\end{listing} 
\section{Descriptive Statistics: Quantitative Data}
\textbf{Describing} the distribution of a quantitative variable involves \textbf{3 aspects}
\begin{enumerate}
  \item \textbf{Shape}
  \item \textbf{Centre} - The middle of the distribution like mean, median, and mode
  \item \textbf{Spread} - The variation or dispersion of the distribution
\end{enumerate}

\subsection{Shape of a Distribution}
Skewness is defined by the shape of the tail. The type of the skew is defined by the tail of the distribution. Longer left tail it's left skewed, and vice versa. 

You should now what the population and sample means are as well as the variance. Showing whiskers in a box plot using IQR and the distance ratio. 

\section{Central Limit Theorem}
You should know this and its magic number of 30 samples.

\section{Hypothesis testing and Determining CI}
You should know how to do this. 


\end{document}
