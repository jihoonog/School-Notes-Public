\documentclass{article}
\usepackage{qtree}
\begin{document}
Lossless-join, Dependency Preserved Decomposition to 3NF.
\begin{itemize}
    \item Given a relation R with a minimal set of FDs.
    \item Find a lossless-join decomposition of the R to BCNF.
    \item For every FD $X \rightarrow A$ which is not preserved after the decomposition, create a new relation with the schema $XA$.
    \item If the two relations $R1(X)$ and $R2(Y)$ exist where $X \subseteq Y$ delete $R1(X)$
\end{itemize}

\section{Example}
FDs:\\
$A \rightarrow B$, $A^+ = AB$\\
$CE \rightarrow D$, $CE^+ = CDEABF$\\
$BC \rightarrow D$, $BC^+ = BCDAB$\\
$AE \rightarrow F$, $AE^+ = AEFB$\\
$CD \rightarrow A$, $CD^+ = CDAB$
\Tree [.R(A,B,C,D,E,F)\\$A\rightarrow B$ R1(A,B)\\$A\rightarrow B$ [.R2(A,C,D,E,F)\\$AE\rightarrow F$ R3(A,E,F)\\$AE\rightarrow F$ [.R4(A,E,C,D)\\$CD\rightarrow A$ R5(C,D,A)\\$CD\rightarrow A$ R6(E,C,D)\\$CE\rightarrow D$ ]]]\\ 
The leaves of the tree are the relations that will create a lossless-join in BCNF.\\
However, because $BC\rightarrow D$ is missing from the FD we create a new relation R7(B,C,D) and use that in conjunction with the relations created from the decomposition to create a lossless-join, dependency perserving decomposition in 3NF.
\pagebreak
\section{Why do we want to preserve the FDs?}
\subsection{Answer}
In order to maintain dependency constraint on the set of relations, one may have to join the table in order to check such constraint.
Joining a table is computationally intensive and may lead to performance problems if the constraint is checked multiple times.
\section{Summary}
\begin{enumerate}
    \item No redundancy (trade off with Dependency preserving)
    \item Minimal number of relations
    \item Lossless join (Necessary)
    \item Dependency preserving (trade off with no redundancy)
\end{enumerate}
\end{document}