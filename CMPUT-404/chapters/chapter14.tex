\documentclass[../CMPUT-404-Notes.tex]{subfiles}
\begin{document}
\chapter{TLS}
Stands for Transport Layer Security, this works on top of TCP/UDP
Formally called SSL, HTTP inside of TLS is called HTTPS.

Security is about 
\begin{itemize}
  \item Identity - Who are you talking to?
  \item Privacy - Who can hear what you're saying?
  \item Authenticity - Is what you're hearing really coming from the person you think you're talking to?
\end{itemize}

\section{Identity}
Without TLS, I don't know who I'm talking to, things can be intercepted by IP, TCP, and DNS attacks. Without TLS it's relatively easy to intercept connections meant for a web server and serve them yourself.

\subsection{Certificates}
To check for identity TLS will use certificate, this works like an ID card. The server sends a certificate with its domain name (called a Common Name) and a big number called a public key. The certificate also includes other details, such as who made it, for example, "Bank Inc."

The certificate's signature is unique to that certificates. If you change the public key or the domain name, the signature won't match anymore.

The certificate's signature also comes from a specific entity that made that signature. 
It tells who signed it, and only they could have produced that signature.
No one else can pretend to have signed the certificate, or the signature won't match.

It uses public-private keys to sign the certificates. You use the private key to sign the certificate and the public key to verify the certificate.

Specifically you use the CA's public key to verify the certificate.

\subsubsection{Issues}
\begin{itemize}
  \item You don't have the CA's public key - There's no way to know if the bank's domain name and public key are correct
  \item The CA is evil - There's no way to know if the bank's domain name and public key are correct 
  \item The CA's private key got leaked - There's no way to know if the bank's domain name and public key are correct
  \item The bank didn't send a signature from a CA - There's no way to know if the bank's domain name and public key are correct
\end{itemize}

Usually these are combatted by the browser makers:
\begin{itemize}
  \item Only allow trusted CAs
  \item Remove bad or leaked CAs
  \item Not allowing you to view sites without a valid signature from a good CA
  \item Checking lists of known certificates to make sure someone didn't get a second certificate to pretend to be the bank (Certificate Transparency)
  \item Checking lists of known bad certificates and certificates that have been "revoked" (CRLs)
  \item Ensuring the public key is the same as the last time you connected to the site (HPKP)
\end{itemize}

\section{Privacy}
Once the identity of the server is established, and you know the server's public key is really for the server you want to be talking to. TLS can provide privacy and authentication. 

\textbf{Privacy} is hiding the content being transited back and forth from eavesdropping, snooping, man-in-the-middle. We want to prevent anyone but us and the server we want to send our information to from being able to read anything exchanged between us and the server.


\section{Authenticity}
Separate from privacy, every message is hashed to ensure it really came from the server (client).
Prevent replay attack,s injection attacks, MITM, everything encrypted is also authenticated.

\section{TLS Handshake}
\begin{enumerate}
  \item TCP Handshake
  \item Client sends TLS "hello" as well as 
  \begin{itemize}
    \item TLS version
    \item Encryption algorithms
    \item ALPN: Application-Layer Protocol Negotiation e.g., select HTTP/2
    \item SNI: Server Name Identification Equivalent to Host: header
  \end{itemize}
  \item Server sends TLS "hello"
  \begin{itemize}
    \item Chosen encryption algorithm
    \item Certificate
  \end{itemize}
  \item Client verifies server certificate using CA
  \item Client and server agree on an encryption key either:
  \begin{itemize}
    \item Client sends encryption key encrypted with server's public key
    \item Client and server agree on encryption key by sending random bytes both directions (Diffie-Hellman Key Exchange) (more common)
  \end{itemize}
  \item Client sends encrypted "ready"
  \item Server sends encrypted "ready"
\end{enumerate}

\section{Perfect Forward Secrecy}
Newer versions of TLS. Data remains private even if the server or client later get hacked.
Uses \emph{Ephemeral Diffie-Hellman Key Exchange} 

\section{Reducing Latency}
Traditional TLS requires at least 2 round trip latency on top of TCP handshake. Newer TLS supports 1 round trip handshake protocol.

\section{Cipher Suites}
Client and server agree on crypto algorithm to use during handshake.
Many different crypto algorithms are available.

\subsection{Example}
\texttt{TLS_ECDHE_ECDSA_WITH_CHACHA20_POLY1305_SHA256} with 256-bit keys, TLS 1.2
\begin{itemize}
  \item Using TLS 1.2 protocol
  \item Using ECDSA (Elliptic Curve Digital Signature Algorithm)
  \item Using ECDHE (Ephemeral Elliptic Curve Diffie-Hellman Exchange) to agree on an encryption key
  \item Using CHACHA20 encryption algorithm with 256-bit key
  \item Using Poly1305 to check authenticity of most messages
  \item Using SHA-256 to check authenticity of some message
\end{itemize}

\section{TLS Problems}
TLS has had lots of different security problems over the years, typically with fun names!
\begin{itemize}
  \item Renegotiation
  \item FREAK
  \item Logjam
  \item DROWN
  \item BEAST
  \item CRIME
  \item BREACH
  \item padding oracle
  \item Lucky Thirteen
  \item POODLE
  \item Old, insecure crypto algorithms
  \item Truncation
  \item Unholy PAC
  \item Sweet32
  \item Heartbleed
  \item BERserk
  \item Cloudbleed 
\end{itemize}

\section{Avoiding Security Problems}
TLS everything
\begin{itemize}
  \item Make sure all traffic to/from your web app is running over TLS
  \item Example: it may not help if only login is over TLS because an attacker can replace your TLS-encrypted login page with an unencrypted one by replacing the link on the unencrypted main site
\end{itemize}

\section{Avoiding TLS security Problems}
\begin{itemize}
  \item Keep software up to date - No Windows XP
  \item Don't communicate with out-of-date software
  \item Example: if something on your site is meant to be secret, even one bad client can leak the secret!
\end{itemize}









\end{document}
