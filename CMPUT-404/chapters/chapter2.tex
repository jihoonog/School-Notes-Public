\documentclass[../CMPUT-404-Notes.tex]{subfiles}
\begin{document}
\chapter{Internet and Web}
\section{Web}
The web is a "tool" that we use to request, search, navigate and share information. We can use this web to access and operate software.
This \emph{tool} is really a collection of interconnected computers that is spread throughout the entire world and even in space. 

You can use Firefox to inspect a website and see the source code that makes up the website.
If you want to look at traffic you can use Wireshark to see the individual ethernet frame and protocol.

The web is \emph{not} the internet. The internet is a network of computers like desktops, laptops, phones, TV, etc. 
While the World Wide Web is a network of pages like Wiki pages, new stories, blog post, etc.
The web is a connection of software or virtual pages, while the internet is a connection of hardware. 
This also means that the web and internet have different domains of connections.
The internet is connected using physical means like ethernet, wifi, cable, satellites, etc.
While the WWW uses protocols and web formats to connect between different websites or pages. Examples include hyperlinks, HTML, images, redirects, etc.
These are more software orientated connections.

\subsection{Primary Components}
You have web servers that serve web content and web clients like web browsers like Chrome or Firefox that connects to web servers.
The connection is done via HTTP (HyperText Transport Protocol) and HTML (HyperText Markup Language). You can also use Hypertext which are text that has links in it to take you somewhere else.

\begin{Definition}
  {Hypertext and Hypermedia}
  By now the word "hypertext" has become generally accepted for branching and responding text, but the corresponding word "hypermedia", meaning complexes of branching and responding graphics, movies and sound - as well as text – is much less used. Instead they use the strange term "interactive multimedia": this is four syllables longer, and does not express the idea of extending hypertext.

  Nelson, Literary Machines, 1992 
\end{Definition}

\end{document}
