\documentclass[../CMPUT-404-Notes.tex]{subfiles}
\begin{document}
\chapter{HTTP Performance}
It's a non functional requirement

They can be measured in terms of 
\begin{itemize}
    \item requests
    \item volume
    \item latency
    \item bandwidth
    \item utilization 
\end{itemize}

\section{Before you Optimize}
\begin{enumerate}
    \item Measure Something
    \item Record the performance metric
    \item Record and track the original settings for that metric
    \item Run the tests multiple times to get a sample. Avoid luck and outliers
\end{enumerate}

Because we are using statistics to quantity and compare performance, we use statistical analysis to quantify and determine if there is a quantifiable difference. 
To determine if there is an actual difference then make a t-test comparing the two performances and see if the difference is significant. 

\section{Web Performance Best Practices}
Google says:
\begin{itemize}
    \item Optimizing caching - keeping your application's data and logic off the
    network altogether
    \item Minimizing round-trip times - reducing the number of serial request response cycles
    \item Minimizing request overhead - reducing upload size
    \item Minimizing payload size - reducing the size of responses, downloads, and cached pages
    \item Optimizing browser rendering - improving the browser's layout of a page
    \item Optimizing for mobileNew! - tuning a site for the characteristics of mobile
    networks and mobile devices
\end{itemize}

\section{Caching}
\begin{itemize}
    \item Caching increase locality
    \item Locality increases bandwidth
    \item Locality decreases latency
    \item Levels of cache:
    \begin{itemize}
        \item CPU
        \item Memory
        \item Disk
        \item Network
    \end{itemize}
\end{itemize}
Use \texttt{ETags} and \texttt{Last-Modified} headers.

\subsection{User Agent Cache}
\begin{quotebox}
    Also, if the response does have a Last-Modified
    time, the heuristic expiration value SHOULD be
    no more than some fraction of the interval since
    that time. A typical setting of this fraction might
    be 10\%. – RFC2616 section 13
\end{quotebox}
This means if it was modified 10 minutes ago, you should probably hit it up again in a minute.

Thus it is up to the browser to emit a request.
They do so upon expiry or last modified time heuristic.
Or the user forces a refresh like CTRL-SHIFT-R or ctrl-shit click on the refresh button.
In browser cache is the most local and high performance cache!

\subsection{Cache-Control}
Generally sent by User Agent.
Indicates how they want to handle this request.
It signals proxies and caches how to handle the request.

\subsubsection{no-cache}
You must revalidate - We didn't give it a time
\begin{itemize}
    \item A 304 response is fine
    \item Forces a request out to the server
    \item max-age=0 means the same thing
\end{itemize}

\subsubsection{no-store}
Don't store anything
\begin{itemize}
    \item Suggests that the results are not-cacheable
    and emphemeral.
    \item Will not act as DRM
\end{itemize}

\subsubsection{max-age=}
Cache-Control: max-age=seconds in a HTTP
response tells the Use-Agent the maximum
age they should let this resource last
\begin{itemize}
    \item Easy to deploy
    \item Cache-Control: max-age=259200 = 3 Days
    \item Benefit: no date math for you!
    \item Benefit: No date formatting!
    \item Disadvantage: Have to predict max-age!
\end{itemize}

\subsection{Header specific caching}
These are the header specific for caching

\subsubsection{Expires}
Expires tells the Use-Agent after which date
they should ask for a new instance of the
resource.
\begin{itemize}
    \item Easiest to deploy
    \item Very simple
    \item Causes lots of problems if set wrong!
\end{itemize}

\subsubsection{If-Modified-Since}
\begin{itemize}
    \item Conditional HTTP Request
    \item Return a 304 if not modified since
    \item If-Modified-Since: Mon, 07 Apr 2014 03:00:05 GMT. Don't send me anything new unless the resource has
    been modified after that time.
    \item If the response is anything but a 200 OK, return a
    normal response instead of the 304
\end{itemize}

\subsubsection{Last-Modified}
\begin{quotebox}
    The Last-Modified entity-header field value is
    often used as a cache validator. In simple
    terms, a cache entry is considered to be valid if
    the entity has not been modified since the LastModified value. - HTTP RFC 2616 Fielding et
    al.
\end{quotebox}
Last-Modified is a date that the resource was last modified
\begin{itemize}
    \item Used for simple caching
    \item Requires the HTTP server to respond
\end{itemize}

\subsubsection{ETag}
What if you cannot guess or estimate the time
that content will be safe?
What if content updates all the time, unpredictably?
What if content updates but all changes aren't
that important:\\
\noindent
- E.g. your age does increase every second but
maybe it isn't important to caching to have your
age updated per each second?

How do you make it?
\begin{itemize}
    \item If strong (exact content) just use a hash like SHA1
    \item If weak then hash some content you think is
    relevant and prefix with \texttt{W/"etagvalue"} to
    indicate it is a weak hash
    \item If hashing is pointless make an etag of actual
    values in plaintext
    \item Keep it short
\end{itemize}

HTTP Response Header
\begin{itemize}
    \item Contains a name or tag indicating the content
    or revision of a resource.
    \begin{itemize}
        \item Is not date related
        \item Tends to be content related
        \item Can be any value
        \item Can use any hash
    \end{itemize}
\end{itemize}
    
\subsubsection{Dangers of ETags}
Cookies part II
\begin{itemize}
    \item Etags allow for vendors (advertisers) to finger print your
    client because your client will send the etags back.
    \item If you deny cookies, you tend to send etags
\end{itemize}

Too much information
\begin{itemize}
    \item Some Etags contain irrelevant information!
    \item What if the browser reboots and the Etags are
    lost?
\end{itemize}

\section{Cross Cutting Concern}
Performance is a cross cutting concern
because it interacts with other functionality:\\
\noindent
- Security
\begin{itemize}
    \item Lack of encryption means global proxying
    \item Authentication can limit caching
    \item Authentication can imply state
\end{itemize}
- State
\begin{itemize}
    \item State can limit caching
    \item State can limit layering
\end{itemize}

\section{Round Trips}
DNS Lookups, Connections, HTTP transactions
\begin{itemize}
    \item Async is fast: Just send it! Who cares when it arrives
    \item Rountrips are synchronous and slow: We must wait for a
    response!
    \item Avoid HTTP Redirects that aren't cacheable
    \item Rewrite Server Side
    \item Avoid too many HTTP hosts
    \item Can you piggyback?
    \item CSS Sprites are often recommended to reduce number of image requests
    \item Avoid CSS imports
\end{itemize}

\subsection{Round Trip Tricks}
Use multiple static content hostnames:
\begin{itemize}
    \item Take a hit in DNS lookup
    \item But improve parallel download performance
    \item Static hosts should not be dynamic and have stable IPs
\end{itemize}

\section{Reduce Request Size}
Giant Cookies - NOOOOO
\begin{itemize}
    \item Giant URIs -- NOOOO
    \item Too many headers? NOOOO
    \item Remember all that networking we went over?
\end{itemize}
Try to fit within the MTU!


\section{Avoid Dynamism and Cookies for Static Content}
For static content, do GETs to get it
\begin{itemize}
    \item For static content avoid dynamism and cookies
    \item Cookies imply state and can mess up caching
    \item Use separate domains for static content to
    \item avoid statefulness
\end{itemize}

\section{Minimize Resource Size}
\begin{itemize}
    \item Images - too big
    \item Javascript - minify (I dislike this one)
    \item GZIP Encoding!
    \item Sound - too big
    \item Video - too big
    \item You can fake Sound and Video in JS!
\end{itemize}

\section{Minimize Number of Resources}
\begin{itemize}
    \item 1 or 0 CSS Files
    \item 1 or 0 Javascript Files
    \item 1 or 0 Images (CSS Sprites)
    \item 1 or 0 HTML Files
\end{itemize}
You could generate a page in JS and take no hit.
1 giant page has the problem if it is dynamic, but if
it is 1 giant page that does dynamic things you can
cache that page and never have to get HTML/JS/CSS
again.

\section{Optimize Rendering}
Recommendations:
\begin{itemize}
    \item CSS at the top
    \item Javascript at the bottom
    \item Content in the middle
    \item Give appropriate sizes and hints
    \item The layouter is quite expensive, give some hints
    and it will go to town.
\end{itemize}

\section{Defer JavaScript}
Nasty trick:
\begin{itemize}
    \item Encode javascript as a string and eval it later
    when you need it.
    \item Avoids heavy page loading and JS parsing with
    the browser is working hard.
    \item CPU driven
\end{itemize}

\section{Content Delivery Network}
\begin{itemize}
    \item These are global networks of content
    providers that can mirror your content.
    \item Excellent latency and locality
    \item Great for static stuff
    \item \$\$\$
\end{itemize}
    

\section{GZIP things}
\begin{itemize}
    \item Turn on GZIP encoding to make things smaller
    \item Watch out! This affects latency
    \item Improves bandwidth
    \item Balancing act
    \item Generally recommended
\end{itemize}

\section{Fix DNS}
\begin{itemize}
    \item Use A and AAAA records instead of CNAMEs
    \item Have your authoritative name server give the results
    \item Allow for caching of DNS requests to avoid excessive lookups
    \item Provide many A records in 1 response
    \item Reduce number of hosts on 1 page
    \item Some recommend using CNAME to allow multiple connections (so it depends!)
\end{itemize} 

\section{Avoid Indirection}
\begin{itemize}
    \item Redirections
    \item Imports
    \item Dynamic choosing of content
    \item Javascript downloading of content
\end{itemize}
    
\section{Avoid POST}
POST is not idempotent
\begin{itemize}
    \item POST is not cacheable
    \item POST is dynamic
    \item POST smashes all the performance
    infrastructure in a fine powder and blows it
    into your face.
\end{itemize}
    
\section{Check for errors}
403s, 404s, 410s, etc. Are all slow
\begin{itemize}
    \item Often the server works harder to serve an error
    than it does to serve real content.
    \item Errors cause exceptions, exceptions cause latency
    and pain and reporting.
    \item Errors cause logging - more IO
    \item Errors are worthless requests hogging up
    resources
\end{itemize}

\section{Encoding}
Sending a JSON encoding is not always
appropriate in terms of size.
\begin{itemize}
    \item Sending an audio stream as ASCII text? Bad idea. Use an appropriate format.
    \item Do you need lossy or losseless?
    \item Do you need XML? JSON? CSV? Binary?
\end{itemize}

\section{Repeat yourself or don't repeat yourself}
Cache matters if the cache can avoid you
repeating yourself then go for it.
But sometimes denormalizing data and
providing duplicate information avoids more
requests.

\section{Async over Sync}
Asynchronous means that you can do other
things while something is occurring.
\begin{itemize}
    \item Synchronous is blocking which implies latency.
    \item Synchronous means round trips
    \item Async means parallelizable
    \item Synchronous means serialized
\end{itemize}

\section{Javascript Includes}
Some people like to include common libraries
from the library homepage.
\begin{itemize}
    \item Benefit: Someone else has done this so the user has cached it.
    \item Disadvantage: What if they get hacked
    \item Disadvantage: What if they are slower than you?
    \item Disadvantage: What if you lose locality?
\end{itemize}

\end{document}