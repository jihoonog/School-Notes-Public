\documentclass[../MATH-2000-Notes.tex]{subfiles}
\begin{document}
\chapter{Proof by Induction}
\section{The Principle of Mathematical Induction}
\begin{Axiom}
    {Principle of Mathematical Induction}
    \label{InductionP(n)}

    Suppose $P(n)$ is a predicate of natural numbers. If
    \begin{enumerate}
        \renewcommand{\theenumi}{\roman{enumi}}
        \renewcommand{\labelenumi}{\rm(\theenumi)}
        \item \label{InductionP(n)-base} $P(1)$ is true,
              and
        \item \label{InductionP(n)-step}
              for every $ k \ge 2, \bigl( P(k-1) \implies P(k) \bigr)$,
    \end{enumerate}
    then $P(n)$ is true for all $n \in \N^+$.
\end{Axiom}
\Cues{pretty much like recursion but upward instead of downwards}
\begin{Definition}
    {Terminology}~\\
    \begin{itemize}
        \item In a proof using Mathematical Induction, establishing~(\ref{InductionP(n)-base}) is called the \textbf{base case}, and establishing~(\ref{InductionP(n)-step}) is the \textbf{induction step}.
        \item In the induction step, we are proving $P(k-1) \implies P(k)$, so we assume that $P(k-1)$ is true (and establish $P(k)$). This assumption $P(k-1)$ is called the \textbf{induction hypothesis}.
    \end{itemize}
\end{Definition}
\begin{Proposition}
    {Sum of Natural Numbers}
    \label{1+...+n}
    For every $n \in \N^+$, we have
    \ $\displaystyle 1 + 2 + 3 + \cdots + n = \frac{n(n+1)}{2} $.
\end{Proposition}

\begin{proof}[Proof by induction]
    Define $P(n)$ to be the assertion
    $$ 1 + 2 + 3 + \cdots + n = \frac{n(n+1)}{2} .$$

    (i) \emph{Base case.} For $n = 1$, we have
    $$1 + 2 + 3 + \cdots + n = 1 \mbox{\quad and\quad}  \frac{n(n+1)}{2}=\frac{1(1+1)}{2}=1.$$
    Since these are equal,
    $P(1)$ is true.

    \medbreak
    (ii) \emph{Induction step.} Assume $P(k-1)$ is true (and $k \ge 2$). This means that
    $$ 1 + 2 + 3 + \cdots + (k-1) = \frac{(k-1)\bigl( (k-1)+1\bigr)}{2} .$$
    Hence
    \begin{align*}
        1 + {} & 2 + 3 + \cdots + k
        \\&= \bigl(1 + 2 + 3 + \cdots + (k-1) \bigr) + k
        \\&= \frac{(k-1)\bigl( (k-1)+1\bigr)}{2} + k 
        \\& \text{(Induction Hypothesis)}
        \\&= \frac{(k-1)k}{2} + k
        \\&= k \left( \frac{k-1}{2} + 1 \right)
        \\&= k \left( \frac{k+1}{2} \right)
        \\&= \frac{k (k+1)}{2}
        ,\end{align*}
    so $P(k)$ is true.

    Therefore, by the Principle of Mathematical Induction, we know $P(n)$ is true for all~$n$. This means
    $$ 1 + 2 + 3 + \cdots + n = \frac{n(n+1)}{2} $$
    for every $n \in \N^+$.
\end{proof}

\begin{Proposition}
    {\(2n^2 + n\)}
    For every $n \in \N^+$, we have
    $$ 3 + 7 + 11 + \cdots + (4n-1) = 2n^2 + n .$$
\end{Proposition}

\begin{proof}[Proof by induction]
    Define $P(n)$ to be the assertion
    $$ 3 + 7 + 11 + \cdots + (4n-1) = 2n^2 + n .$$

    (i) \emph{Base case.} For $n = 1$, we have
    $$3 + 7 + 11 + \cdots + (4n-1) = 3 \mbox{\quad and\quad}  2n^2 + n = 2(1^2) + 1 = 3.$$
    Since these are equal,
    $P(1)$ is true.

    \medbreak
    (ii) \emph{Induction step.} Assume $P(k-1)$ is true (and $k \ge 2$). This means that
    $$ 3 + 7 + 11 + \cdots + \bigl(4(k-1)-1 \bigr) = 2(k-1)^2 + (k-1) .$$
    Hence
    \begin{align*}
        3 +  {} & 7 + 11 + \cdots + (4k-1)
        \\&= \Bigl( 3 +  7 + 11 + \cdots + \bigl(4(k-1)-1\bigr) \Bigr) + (4k-1)
        \\&= \Bigl( 2(k-1)^2 + (k-1) \Bigr)  + (4k-1)
        \\& \text{(Induction Hypothesis)}
        \\&= \Bigl( 2(k^2 - 2k + 1) + (k-1) \Bigr)  + (4k-1)
        \\&= (2k^2 - 4k + 2) + (k-1)  + (4k-1)
        \\&= 2k^2 + k
        ,\end{align*}
    so $P(k)$ is true.

    \medbreak

    Therefore, by the Principle of Mathematical Induction, we know $P(n)$ is true for all~$n$. This means
    $$ 3 + 7 + 11 + \cdots + (4n-1) = 2n^2 + n $$
    for every $n \in \N^+$.
\end{proof}
\newpage
\section{Other proofs by induction}
\begin{commentbox}{Example on Mondulo Operations}[{PhbLightCyan}]
    Suppose $a,b,n \in \Z$, with $a \equiv b \pmod{n}$. Show
    $a^k \equiv b^k \pmod{n}$, for all $k \in \N^+$.
\end{commentbox}

\begin{proof}[Proof by induction]
    We induct on~$k$. Define $P(k)$ to be the assertion
    $$ a^k \equiv b^k \pmod{n}.$$

    (i) \emph{Base case.} Since $a^1 = a$ and $b^1 = b$, the hypothesis $a \equiv b \pmod{n}$ tells us that
    $$ a^1 \equiv b^1 \pmod{n} ,$$
    so $P(1)$ is true.

    (ii) \emph{Induction step.} Assume $P(k-1)$ is true.
    This means that
    $$a^{k-1} \equiv b^{k-1} \pmod{n} .$$
    By assumption, we also have
    $$ a \equiv b \pmod{n} .$$
    We can multiply the above congruences, to conclude that
    $$ (a^{k-1} )(a) \equiv ( b^{k-1}) (b) \pmod{n} .$$
    In other words,
    $$ a^k \equiv b^k \pmod{n} ,$$
    so $P(k)$ is true.

    \medskip
    Therefore, by the Principle of Mathematical Induction, $P(k)$ is true for every $k \in \N^+$.
\end{proof}

\begin{Definition}
    {Fibonacci Numbers}
    The \textbf{Fibonacci numbers} $F_1,F_2,F_3,\ldots$ are defined by:
    \begin{itemize}
        \item $F_1 = 1$,
        \item $F_2 = 1$,
              and
        \item $F_n = F_{n-1} + F_{n-2}$ for $n \ge 3$.
    \end{itemize}
    (For example, $F_3 = F_{3-1} + F_{3-2} = F_2 + F_1 = 1 + 1 = 2$.)
    In general, each Fibonacci number (after~$F_2$) is the sum of the two preceding Fibonacci numbers, so
    the first few Fibonacci numbers are:
    $$\begin{array}{c||c|c|c|c|c|c|c|c}
            n   & 1 & 2 & 3 & 4 & 5 & 6 & 7  & \cdots \\
            \noalign{\hrule}
            F_n & 1 & 1 & 2 & 3 & 5 & 8 & 13 & \cdots
        \end{array}$$
\end{Definition}

\begin{commentbox}{Example on Fibonacci Numbers}[{PhbLightCyan}]
    Prove \ $\displaystyle \sum\limits_{k=1}^n F_k = F_{n+2} - 1$ \ for all $n \in \N^+$.
\end{commentbox}
\begin{proof}[Proof by induction]
    Define $P(n)$ to be the assertion
    $$ \sum_{k=1}^n F_k = F_{n+2} - 1 .$$

    \medskip

    (i) \emph{Base case.} For $n = 1$, we have
    $$ 
    \begin{aligned}
        \sum_{k=1}^n F_k = \sum_{k=1}^1 F_k &+= F_1 = 1 = 2 - 1 = F_3 - 1 
        \\&= F_{1+2} - 1 = F_{n+2} - 1,
    \end{aligned}
    $$
    so $P(1)$ is true.

    \medskip
    (ii) \emph{Induction step.} Assume $P(n-1)$ is true (and $n \ge 2$). Then
    \begin{align*}
        \sum_{k=1}^n F_k
         & = \left( \sum_{k=1}^{n-1} F_k \right) + F_n
        \\&= \Bigl( F_{(n-1)+2} - 1 \Bigr) + F_n && \text{(Induction Hypothesis)}
        \\&= (F_{n+1} - 1) + F_n
        \\&= (F_{n+1} + F_n) - 1
        \\&= F_{n+2} - 1
         &                                             & \begin{pmatrix} \text{definition of} \\ \text{Fibonacci number} \end{pmatrix}
        .\end{align*}

    \medskip
    Therefore, by the Principle of Mathematical Induction, $P(n)$ is true for every~$n$. This means
    $\displaystyle \sum\limits_{k=1}^n F_k = F_{n+2} - 1$ for all $n \in \natural^+$.
\end{proof}
\newpage
\section{Other versions of induction}
There are other versions of induction that works better in solving other cases.
\begin{Proposition}
    {}
    \label{pr-OtherInduct}
    \renewcommand{\theenumii}{\roman{enumii}}
    \makeatletter\renewcommand{\p@enumii}{}\makeatother
    Suppose $P(n)$ is a predicate of natural numbers, and $m \in \N^+$.
    \begin{enumerate}
        \item \label{pr-OtherInduct-strong}
              {\rm(Strong induction)}
              If
              \begin{enumerate}
                  \item \label{InductionStrong-base} $P(1)$ is true,
                        and
                  \item \label{InductionStrong-step}
                        for every $ n \ge 2$,
                        \\ \hbox{\qquad} $\Bigl( \bigl( \mbox{for every } k \in \{1,2,\ldots,n-1\}, P(k) \bigr) \implies P(n) \Bigr)$,
              \end{enumerate}
              then $P(n)$ is true for all $n \in \N^+$.

        \item \label{pr-OtherInduct-Gen}
              {\rm(Generalized induction)}
              If
              \begin{enumerate}
                  \item \label{InductionGen-base} $P(m)$ is true,
                        and
                  \item \label{InductionGen-step}
                        for every $ n > m$, $\bigl( P(n-1)  \implies P(n) \bigr)$,
              \end{enumerate}
              then $P(n)$ is true for all $n \ge m$.

        \item \label{pr-OtherInduct-multiple}
              {\rm(Strong induction with multiple base cases)}
              If
              \begin{enumerate}
                  \item \label{InductionMultiple-base} $P(k)$ is true for all $k \in \{1,2,\ldots,m\}$,
                        and
                  \item \label{InductionMultiple-step}
                        for every $ n > m$,
                        \\ \hbox{\qquad} $\Bigl( \bigl( \mbox{for every } k \in \{1,2,\ldots,n-1\}, P(k) \bigr) \implies P(n) \Bigr)$,
              \end{enumerate}
              then $P(n)$ is true for all $n \in \N^+$.

        \item \label{pr-OtherInduct-n+1}
              If
              \begin{enumerate}
                  \item \label{pr-OtherInduct-n+1-base}
                        $P(1)$ is true,
                        and
                  \item \label{pr-OtherInduct-n+1-step}
                        for every $ k \in \N^+$, $P(k) \Rightarrow P(k+1)$,
              \end{enumerate}
              then $P(n)$ is true for all $n \in \N^+$.

        \item \label{pr-OtherInduct-set}  % was \label{InductionSet}
              Suppose $S \subset \N^+$. If
              \begin{enumerate} \renewcommand{\theenumi}{\roman{enumi}}
                  \item \label{InductionSet-base} $1 \in S$,
                        and
                  \item \label{InductionSet-step}
                        for every $n \in S, \bigl( n+1 \in S \bigr)$,
              \end{enumerate}
              then $S = \N^+$.

    \end{enumerate}
\end{Proposition}
\Cues{Strong induction goes down to the base case}
\begin{commentbox}{Example}[{PhbLightCyan}]
    \label{Fn<2nEg}
    Prove $F_n < 2^n$, for every $n \in \N^+$.
\end{commentbox}
\begin{proof}[Proof by induction]
    Define $P(n)$ to be the assertion
    $$ F_n < 2^n .$$
    We use strong induction with $2$~base cases.

    (i) \emph{Base cases.}
    We have
    $$ F_1 = 1 < 2 = 2^1 ,$$
    and
    $$ F_2 =1 < 4 = 2^2 ,$$
    so $P(1)$ and $P(2)$ are true.

    (ii) \emph{Induction step.} Assume $n \ge 3$, and that $P(n-1)$ and $P(n-2)$ are true. We have
    \begin{align*} F_n
         & = F_{n-1} + F_{n-2}
        \\&< 2^{n-1} + 2^{n-2} && \text{(Induction Hypotheses)}
        \\&< 2^{n-1} + 2^{n-1}
        \\&= 2^n
        ,\end{align*}
    so $P(n)$ is true.

    By the Principle of Mathematical Induction (in the form of strong induction with multiple base cases), we conclude that $P(n)$ is true for all $n \in \N^+$.
\end{proof}

\section{Well-ordered}
\begin{Definition}
    {Smallest}
    Let $S \subset \real$ and $a \in \real$. We say $a$ is the \textbf{smallest} element of~$S$ if and only if:
	\begin{itemize}
	\item $a \in S$,
	and
	\item $\forall s \in S$, $a \le s$.
	\end{itemize}
\end{Definition}

\begin{Theorem}
    {\(\N\) is well-ordered}\label{NisWO}
    Every nonempty subset of~$\N$ has a smallest element.
\end{Theorem}
\begin{Note}
    If \(P(n)\) can be proven for all \(n\in\N \) using induction, then it can also be proven by applying Theorem \ref{NisWO} ot the set \[ S = \{\, n \in \N^+ \mid \neg P(n) \,\} .\]
\end{Note}

\section{Application to Number Theory}
\begin{Definition}
    {Prime}
    An element~$p$ of~$\N^+$ is \textbf{prime} if and only if $p > 1$ and $p$ is not divisible by any element of~$\N^+$ other than~$1$ and~$p$.
\end{Definition}

\begin{Proposition}{} \label{HasPrimeFactor}
    If $n \in \N$ and $n > 1$, then $n$~is divisible by a prime number.
\end{Proposition}
\begin{proof}
    Suppose there is some natural number $n > 1$, such that $n$~is not divisible by a prime number. (This will lead to a contradiction.) Since $\N$~is well-ordered, we may assume that $n$~is the smallest such number, so:
    
    If $1 < k < n$ (and $k \in \N$), then $k$~is divisible by a prime number.
    
    Since $n \mid n$, but (by assumption) $n$~is not divisible by any prime number, we know that $n$~is not prime. By definition, this means there exists $k \in \N$, such that $k \mid n$ and $1 < k < n$. From the minimality of~$n$, we know that $k$~is divisible by some prime number~$p$. Then $p \mid k$ and $k \mid n$, so $p \mid n$. This contradicts the fact that $n$~is not divisible by a prime number.
\end{proof}

\begin{Theorem}{Fundamental Theorem of Arithmetic}
    Every natural number \textup(other than $0$ and~$1$\textup) is a product of prime numbers \textup(or is itself a prime\textup).
\end{Theorem}
    
\begin{proof}[Proof by contradiction]
    Suppose there is some natural number $n > 1$, such that $n$~is not a product of prime numbers (and is not a prime). Since $\N$~is well-ordered, we may assume that $n$~is the smallest such number, so:
    
    If $1 < k < n$ (and $k \in \N$), then $k$~is a product of prime numbers.
    
    Since $n$~is not prime, it is divisible by some natural number~$k$, with $1 < k < n$. This means we may write $n = km$, for some $m \in \N^+$. Since $m = n/k$ and $1 < k < n$, we see that $1 < m < n$. Therefore, the minimality of~$n$ implies that $k$ and~$m$ are products of prime numbers: say $k = p_1 p_2 \cdots p_r$ and $m = q_1 q_2 \cdots q_s$. Then
        $$ n = km = (p_1 p_2 \cdots p_r)(q_1 q_2 \cdots q_s) $$
    is a product of prime numbers. This is a contradiction.
\end{proof}


\begin{Corollary}
    {Infinite many primes}
    There are infinitely many prime numbers
\end{Corollary}


\begin{proof}[Proof by contradiction]
    Suppose there are only finitely many prime numbers. Then we can make a list of all of them: 
        $$ \text{The set of all prime numbers is $\{p_1,p_2,\ldots,p_n\}$.} $$
    Let 
        $$ N = p_1\times p_2 \times \cdots \times p_n .$$
    From \ref{HasPrimeFactor}, we know there is some prime~$p$, such that $p \divides (N+1)$. 
    
    Since $p_1,p_2,\ldots,p_n$ is a list of all the prime numbers, we know $p = p_i$, for some~$i$. Therefore $p = p_i$ is one of the factors in the product that defines~$N$, so $p \divides N$. Therefore, $p$ divides both $N$ and~$N+1$, so we have
        $$ p \divides \ \bigl( (N+1) - N\bigr) \ = \ 1 .$$
    This implies $p = \pm 1$, which contradicts the fact that~$p$, being a prime number, must be $> 1$.
\end{proof}


\begin{Definition}
    {relatively prime}
    Let $a,b \in \N^+$. We say $a$ and~$b$ are \textbf{relatively prime} if and only if they have no divisors in common, other than~$1$. (I.e., if $k \in  \N^+$, and $k$ is a divisor of both $a$ and~$b$, then $k = 1$. In other words, the "greatest common divisor" of $a$ and~$b$ is~$1$.)
\end{Definition}

\begin{Theorem}
    {}
    \label{GCDLinComb}
    Let $a,b \in \N^+$. If $a$ and~$b$ are relatively prime, then there exist $m,n \in \Z$, such that $ma + nb = 1$.
\end{Theorem}

\begin{proof}
    Let
        $$ S = \{\, ma + nb \mid m,n \in \Z\ k =  \,\} \cap \N^+. $$
    It is easy to see that $a \in S$ (by letting $m = 1$ and $n = 0$), so $S \neq \emptyset$. Therefore, since $\N$ is well-ordered, we may let $d$ be the smallest element of~$S$.
    Then $d \in S$, so we have $d = m_0 a + n_0 b$ for some $m_0,n_0 \in \Z$.
    
    By the Division Algorithm \ref{DivAlgThm}, we may write 
        $$ \text{$a = qd + r$ with $0 \le r < d$} .$$
    So
        $$ r = a - qd = a - q(m_0a+n_0b) = (1-qm_0)a + qn_0b = m a + n b ,$$
    where $m = 1 - qm \in \Z$ and $n = qn \in \Z$.
    On the other hand, since $r < d$, and $d$~is the smallest element of~$S$, we know $r \notin S$. 
     From the definition of~$S$, we conclude that $r = 0$. So $d \divides a$.
    
    By repeating the same argument with $a$ and~$b$ interchanged (and $m_0$ and~$n_0$ also interchanged) we see that $d \divides b$. 
    
    Therefore, $d$ is a divisor of both~$a$ and~$b$. Since $a$ and~$b$ are relatively prime, we conclude that $d = 1$. Since $d \in S$, this means $1 \in S$, which establishes the desired conclusion.
\end{proof}
\end{document}