\documentclass[../ECON-281-Notes.tex]{subfiles}

\begin{document}
\chapter{Demand and Supply Analysis}
In this chapter we will review demand and supply, market equilibrium shifts in D and S, and elasticities.

\section{Demand}
The demand curve is a function like so:
\[ 
  Q \cdot d_x = f(P_x, P_{y}, I, tastes, \# buyers, taxes, subsidiaries) 
\] 
Where:
{\centering
\begin{DndTable}[color=PhbLightGreen]{XXX}
  \textbf{Variable} & \textbf{Represent} & \textbf{Relation to $Q\cdot d_x$} \\
  $P_x$ & Price of $x$ & Negative \\
  $P_y$ & Price of related goods & Negative if complement, Positive if substitute \\
  $I $ & Income & Positive if Normal, Negative if Inferior \\
  $tastes$ & Buyers taste & Positive \\
  $\# buyers$ & Numbers of buyers in the market & Positive \\
  $taxes$ & Government taxes & Negative \\
  $subsidiaries$ & Government Subsidiaries & Positive \\    
\end{DndTable}} 
Only $P_x$ will movement along the $!\cdot d_x$ curve, all other inputs will shift the curve left or right if the relation is negative or positive respectively.

To study the effect of one particular factor on $Q\cdot d$ we need to keep other factors constant "ceterius paribus" to isolate their effect.

\begin{Definition}
  {Choke Price}
  Is the price where the quantity demanded is 0. Basically it is the y-intercept of the demand curve.
\end{Definition}

\begin{DndSidebar}[color=PhbLightGreen]{Law of Demand}
  There is a \textbf{negative} relation between $P$ and $Q\cdot d$, assuming that all other factors are kept constant.
\end{DndSidebar}
A change in $Q\cdot d$ is a movement from one point to another point on the same demand curve.
This is only caused by the change in the product's own price.
A change in demand is the shift of the whole demand curve either by shifting left or right for a decrease or increase of the demand respectively. This is caused by the change in other factors other than the product's own price.

Demand can be used for two different sources
\begin{enumerate}
  \item \textbf{Direct} - We demand the good to use it directly.
  \item \textbf{Derived} - We demand the good to use it to produce something else as an intermediate product like we buy paper to make a book.
\end{enumerate}

\subsection{Market Demand}
This is the demand for all consumers the formula can be written like so
\[ 
  Q_m(P) = \sum_{i \in C}^{} Q_i(P) 
\] 
Where 
{\centering
\begin{DndTable}[color=PhbLightGreen]{XX}
  \textbf{Variable} & \textbf{What they mean} \\
  $Q_m$ & Quantity demand for the market \\
  $P$ & Price \\
  $C$ & Set of all consumers \\
\end{DndTable}}

\section{Supply}
The Supply curve is a function like so:
\[
  \begin{aligned}
    Q\cdot S_x &= f(P_x, P_{inputs}, Technology, Taxes, Subsidiaries,\\
    &\# sellers, Weather, P_e) 
  \end{aligned} 
\] 
Where
{\centering
\begin{DndTable}[color=PhbLightGreen]{XXX}
  \textbf{Variable} & \textbf{Represent} & \textbf{Relation} \\
   $P_x$ & Price of $x$  & Positive \\
   $P_{inputs}$ & The price for input to create $x$  & Negative  \\
   $Technology$& The technology used to produce $x$  & Positive \\
   $Taxes$ & Taxes on $x$  & Negative  \\
  $Subsidiaries$ & Subsidiaries on $x$  & Positive  \\
   $\# sellers$&The number of sellers  & Positive  \\
   $Weather$ & The weather to produce the product if affected  & Positive  \\
   $P_{e}$ & The expected future price of $x$   & Negative  \\
\end{DndTable}}
Only $P_{x}$ will causes a movement  along the supply curve or the quality supplied.
All the other factors will shift the supply curve left or right for a decrease or increase in supply respectively.

\section{Equilibrium}
In a free market with no government intervention the equilibrium price will be the intersection where the supply and demand curves meet.
If $q\cdot d < Q\cdot s$ then there is an excess of supply. If $Q\cdot  d > Q\cdot  s$ then there is a shortage of supply. 
The equilibrium of Price and Quantity are caused by endogenous factors while factors that shift the curves are caused by exogenous factors.

\begin{DndSidebar}[color=PhbLightGreen]{Comparative Static analysis }
  What happens to P and Q "endogenous factors" when there is a shift in D or S due to exogenous factors.
\end{DndSidebar}

\section{Shifts in P and Q}
Any shift in D or S will affect both \textbf{P} and \textbf{Q}.
You can do it yourself to check.
However, if both P and Q shift then any one of three things can happen to one of the output factors (P and Q).
\begin{itemize}
  \item The affect value increases
  \item The affect value decreases
  \item The affect value stays the same  
\end{itemize}
In general a shift in both curve will guarantee a movement of one value but an indeterminate movement in the other.

\section{Elasticity}
\textbf{Elasticity} is a measure of sensitivity of the response of quality demanded from a change in price.
There are three types of elasticity for a product, they are:
\begin{enumerate}
  \item Price elasticity of demand
  \item Cross elasticity of demand
  \item income elasticity of demand
\end{enumerate}
\newpage
\subsection{Price elasticity of demand}
\begin{DndSidebar}[color=PhbLightGreen]{Price elasticity of demand}
  It measures how sensitive the response of $q\cdot d$ to the change in the product's own price, other factors kept constant.
  \[ 
    \varepsilon_{Q\cdot p} = \frac{\% \Delta Q\cdot  d}{\% \Delta P} = \frac{\Delta Q}{\Delta P} * \frac{P}{Q}        
  \] 
  For the point formula or
  \[ 
    \frac{\Delta Q}{\Delta P} * \frac{P_1 + P_2}{Q_1 + Q_2} 
  \] 
  For the arc or average formula.  
  \\~\\
  \begin{itemize}
    \item You use the point formula at a point on the supply or demand curve e.g., equilibrium point.
    \item You use the arc or average formula if you are using two points and want to find the elasticity between those two points. $\Delta Q$ and $\Delta P$ are found from the table and not from the equation necessarily.
  \end{itemize}
  
\end{DndSidebar}

\subsection{Income elasticity of demand }
\begin{DndSidebar}[color=PhbLightGreen]{Income elasticity of demand}
  It measures how sensitive the response of $q\cdot d$ to changes in income, other factors kept constant.
  \[ 
  \varepsilon_{I} = \frac{\% \Delta Q\cdot d_x}{\% \Delta I} = \frac{\Delta Q}{\Delta I} * \frac{I}{Q} 
  \] 
  For the point formula or
  \[ 
  \frac{\Delta Q}{\Delta I} * \frac{I_1+I_2}{Q_1+Q_2} 
  \] 
  For the arc or average formula. 

  If the sign is \textbf{negative} then the good is inferior, and if the sign is \textbf{positive} then the good is normal.
\end{DndSidebar}

\subsection{Cross elasticity of demand}
\begin{DndSidebar}[color=PhbLightGreen]{Cross elasticity of demand}
  It measures how sensitive the response of $q\cdot d$ to the change in the price of another related good, other factors kept constant.
  \[ 
  \varepsilon_{y} = \frac{\%\Delta Q_x}{\%\Delta P_y} = \frac{\Delta Q_x}{\Delta P_y} * \frac{P_y}{Q_x} 
  \]
  For the point formula or 
  \[ 
    \frac{\Delta Q_x}{\Delta P_y} * \frac{P_{1y} + P_{2y}}{Q_{1x} + Q_{2x}} 
  \] 
  For the arch or average formula.
  
  If the sign is \textbf{positive} then the related good is a substitute and if the sign is \textbf{negative} then the related good is a complement, if the sign is $0$ then the good is unrelated.
\end{DndSidebar}

  
\subsection{Constant price elasticity of demand}
A demand curve with a constant $\varepsilon_{Q}$ then the demand curve has the following equation:
\[ 
Q = \frac{A}{P^{b}} 
\] 
Where $A$ and $b$ are constants.

\subsection{Factors that determines elasticity for demand}
There are multiple factors that affects the elasticity for demand.
\begin{enumerate}
  \item \textbf{Time period}: In the short run you have less time to find cheaper alternatives so the elasticity is much lower compared to the long run.
  \item \textbf{Availability of substitutes}: If the goods have many substitutes then it is more elastic then if a good has less substitutes.
  \item \textbf{Price of the product (Assume straight demand curve)}: The higher the price of the product the more elastic it is because a larger portion of income is spent on the good.
  \item \textbf{Fraction of income spent}: If a small fraction of income is spent on the good then it is more inelastic if compared to spending a lot more of the income on the good.
  \item \textbf{Definition of the product}: Demand for broadly defined products is less elastic than narrowly defined items because of the available substitutes. 
\end{enumerate}
\begin{Note}
  The reason why price of the product will affect elasticity is because the second term $\frac{P}{Q}$ will increase while the slope of the demand curve is fixed ($\frac{\Delta Q}{\Delta P}$).
  Increasing the price will increase the elasticity.
\end{Note}
\newpage
\subsection{Price elasticity of supply }
\begin{DndSidebar}[color=PhbLightGreen]{Price elasticity of supply}
  It measures how sensitive the response $Q\cdot s$ to the change in the product's own price, other factors kept constant.
\[ 
    \varepsilon_{s} = \frac{\% \Delta Q}{\% \Delta P} = \frac{\Delta Q}{\Delta P} * \frac{P}{Q}        
  \] 
  For the point formula or
  \[ 
    \frac{\Delta Q}{\Delta P} * \frac{P_1 + P_2}{Q_1 + Q_2} 
  \] 
  For the arc or average formula.
 
  The sign for this elasticity is always positive unlike the demand elasticity for price. 
\end{DndSidebar}


\subsection{The Range of elasticity}
The range of elasticity is $\varepsilon  \in \mathbb{R}^+$ i.e., $[0, \infty]$.
Where
{\centering
\begin{DndTable}[color=PhbLightGreen]{XX}
  \textbf{Range} & \textbf{Elasticity} \\
  $0$ & Perfectly Inelastic \\
  $0 < \varepsilon 1$ & Inelastic \\
  $1$ & Unit Elastic \\
  $1 < \varepsilon \infty$ & Elastic \\
  $\infty$ & Perfectly Elastic \\
\end{DndTable}}

\subsection{Relationship between P, Total Revenue and Elasticity of Demand}
With regards to total revenue and elasticity there are five possible scenarios of total revenue:
{\centering
\begin{DndTable}[color=PhbLightGreen]{XX}
  \textbf{Elasticity} & \textbf{Change in Total Revenue} \\
  Perfectly Inelastic & Positive relation between P and TR, the \% change in TR is the same as P\\
  Inelastic  & Positive relation between P and TR , less of an increase in TR as in the Perfectly Inelastic case but still an increase\\
  Unit Elastic & No change in TR for either case because a change in P causes an equal and opposite change in Q \\
  Elastic & Negative relation between P and TR  \\
  Perfectly Elastic & Negative relation between P and TR, any change in P leads TR to 0\\
\end{DndTable}}
In general a smart seller should increase price if the product is inelastic while decreasing the price if the product is elastic. 



\end{document}
